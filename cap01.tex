%=======================================================================
% Escrevendo o Texto
%=======================================================================
\chapter{Escrevendo o Texto}

\section{Comandos do \LaTeX}
Como regra geral, use os comandos tradicionais do \LaTeX\ para formatar seu texto.  Neste documento procuramos demonstrar os comandos mais comumente utilizados em monografias acadêmicas.

Neste capítulo apresentamos alguns exemplos de como colocar figuras e tabelas no seu texto.

\section{Ilustrações}

\subsection{Legendas}
As legendas das figuras devem se encontrar no topo da figura e não abaixo, como usualmente colocado. Abaixo da figura, é obrigatório colocar a fonte (mesmo que a figura tenha sido do próprio autor).

As legendas devem conter o tipo da ilustração (Figura, Tabela, etc), seguido de numeração simples (sem número do capítulo).

Toda figura deve ser citada no texto, como nos exemplos que seguem.

\subsection{Figuras}
A Figura~\ref{fig:escrita} ilustra as fases psicológicas da escrita da dissertação. Você vai se reconhecer no personagem. ;-)

\begin{figure}
	\caption{Fases psicológicas da escrita da dissertação}
	\label{fig:escrita}
	\centering%
	\begin{minipage}{.8\textwidth}
		\includegraphics[width=\textwidth]{escrita}
		\fonte{\citetexto{Cham12}}
	\end{minipage}
\end{figure}

\subsection{Tabelas}
A Tabela~\ref{tab:estacoes} é um exemplo de tabela elaborada pelo(a) próprio(a) autor(a).

\begin{table}
	\caption{Período das estações do ano no Brasil}
	\label{tab:estacoes}
	\centering%
	\begin{minipage}{.6\textwidth}
		\begin{tabular*}{\textwidth}{ll}
			\hline
			\textbf{Meses} & \textbf{Estações do Ano}\\
			\hline
			21 de março a 21 de junho & Outono\\
			21 de junho a 23 de setembro & Inverno\\
			23 de setembro a 21 de dezembro & Primavera\\
			21 de dezembro a 21 de março & Verão\\
			\hline
		\end{tabular*}
		\fonte{Elaborada pela autora.}
	\end{minipage}
\end{table}

\section{Resumo}
O resumo deve conter de 100 a 500 palavras. No resumo não deve haver citações e indica-se que essa seja a última seção do texto a ser escrita. Veja abaixo uma sugestão de organização e exemplo de resumo de \citetexto{Moro11}.

Sugestão (uma a três linhas para cada item):
\begin{itemize}
	\item Contexto geral e específico;
	\item Questão/problema sendo investigado (propósito do trabalho);
	\item Estado-da-arte (por que precisa de uma solução nova/melhor);
	\item Solução (nome da proposta, metodologia básica sem detalhes, quais características respondem as questões iniciais, interpretação dos resultados, conclusões).
\end{itemize}

Exemplo (SANTOS et al., 2008 apud \citealp{Moro11}):
\begin{quote}
CONTEXTO: A Web é abundante em páginas que armazenam  dados de forma implícita. PROBLEMA: Em muitos casos, estes dados estão presentes em textos semiestruturados sem a presença de delimitadores explícitos e organizados em uma estrutura também implícita. SOLUÇÃO: Este artigo apresenta uma nova abordagem para extração em textos semi-estruturados baseada em Modelos de Markov Ocultos (Hidden Markov Models - HMM). ESTADO-DA-ARTE e MÉTODO PROPOSTO: Ao contrário de outros trabalhos baseados em HMM, a abordagem proposta dá ênfase à extração de metadados, além dos dados propriamente ditos. Esta abordagem consiste no uso de uma estrutura aninhada de HMMs, onde um HMM principal identifica os atributos no texto e HMMs internos, um para cada atributo, identificam os dados e metadados. Os HMMs são gerados a partir de um treinamento com uma fração de amostras da base a ser extraída. RESULTADOS: Os experimentos realizados com anúncios de classificados retirados da Web mostram que o processo de extração alcança qualidade acima de 0,97 com a medida F, mesmo se esta fração de treinamento é pequena. 
\end{quote}