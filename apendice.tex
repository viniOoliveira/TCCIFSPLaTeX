%=======================================================================
% Exemplo de Apêndice
% O Apêndice é utilizado para apresentar material complementar elaborado
% pelo próprio autor.  Deve seguir as mesmas regras de formatação do
% corpo principal do documento.
%=======================================================================
\appendix
\chapter{Informações Complementares}

O Apêndice é o lugar para incluir textos complementares, que não são essenciais para o entendimento do assunto principal da monografia, mas que podem contribuir com informação relevante (por exemplo, uma prova matemática, uma conceituação básica, etc.).  Ele deve seguir o formato normal do documento.

%=======================================================================
% Exemplo de Anexo
% O Anexo é utilizado para a ``inclusão de materiais não elaborados pelo
% próprio autor, como cópias de artigos, manuais, folders, balancetes, etc.
% e não precisam estar em conformidade com o modelo''.
%=======================================================================
\annex
\chapter{Artigos Publicados}
Existe diferença entre os Apêndices e os Anexos.  Os apêndices trazem informação escrita pelo próprio autor do trabalho, incorporando-se ao formato da monografia como um todo.  Já um anexo é um material à parte, definido/publicado por si só, e que o autor julga conveniente ser apresentado juntamente com a monografia.  Normalmente também vai apresentar formato próprio, como um artigo publicado, um folder, uma planilha, etc.